\documentclass{article}
\usepackage{amsmath}
\usepackage{amssymb}
\usepackage[top=2cm,bottom=2cm]{geometry}
\usepackage{enumerate}
\usepackage{tikz}
\usepackage{amsfonts}
\title{Homework 10}
\author{Chengyu Lin\footnote{F1003028-5100309007}}
\date{}
\begin{document}
\maketitle

\begin{itemize}
    \item[Problem 1]
        Let $\mathcal{B} = \{(A, x) | A \in \mathcal{A}, x \in A\}$.

        There is a way to count $|\mathcal{B}|$: First choose one
        $\bar{A} \in \partial \mathcal{A}$, then choose $x$ from
        $[n] \setminus \bar{A}$, $(\bar{A} \cap \{x\}, x)$ is what we
        want. Clearly every entries from $\mathcal{B}$ would be count
        at least once.

        Therefore $\partial \mathcal{A} (n - r) \geq \mathcal{A} (r + 1)$.

        ${n \choose r+1} = \frac{{n \choose r} (n - r)}{r + 1}$.

        So

        $$\frac{\partial \mathcal{A}}{{n \choose r}} \geq 
        \frac{\mathcal{A}}{{n \choose r+1}}$$

    \item[Problem 2]
        \begin{enumerate}[(a)]
            \item
                Chain1 : $\emptyset, \{1\}, \{1,2\}, \{1,2,3\}, \{1,2,3,4\}$

                Chain2 : $\{2\}, \{2,3\}, \{2,3,4\}$

                Chain3 : $\{3\}, \{1,3\}, \{1,3,4\}$

                Chain4 : $\{4\}, \{1,4\}, \{1,2,4\}$

                Chain5 : $\{2, 4\}$

                Chain6 : $\{3, 4\}$

            \item
                For $2^{[n]}$, construct the graph with $V = 2^{[n]}$.
                There is an edge between $A$ and $B$ ($WLOG, |A| \le |B|$)
                iff $\exists x, A \cup \{x\} = B$.

                Consider the subgraph induced by ${[n] \choose r}$
                and ${[n] \choose r+1}$. The vertex in the former has
                a degree of $n - r$ and in the latter has a degree
                of $r + 1$. The vertices in smaller side has a higher
                degree. Which means there is always a perfect matching
                between them (number of matches equal to the size of 
                smaller side).

                And ${n \choose r} = {n \choose n - r}$, each time we
                can pick one longest chain start at the very beginning
                (lies in ${[n] \choose k}$) and end at the last
                (lies in ${[n] \choose n-k}$). Which satisfies the 
                condition.
        \end{enumerate}

    \item[Problem 3]
        \begin{enumerate}[(a)]
            \item $${10 \choose 5} = 252 > 200$$
            \item $${9 \choose 4} = 126 > 100$$
        \end{enumerate}

    \item[Problem 4]
        Create a new matrix $M_i$ for each line. The entry at row $j$
        column $k$ represents that line $i$ has a number $k$ set at
        column $j$ in the original matrix $A$.

        Since there is no line contains the same number twice. $M_i$
        is a permutation matrix. Let $M = \sum_{i=1}^{k} M_i$,
        $M$ has $n - k$ 0s in each row and column.

        Let $C$ be a matrix with all entries set to $1$.
        Then $C - M$ can be factored into the sum of
        $n - k$ permutation matrices which implies the result.

\end{itemize}

\end{document}
