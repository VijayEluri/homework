\documentclass{article}
\usepackage{amsmath}
\usepackage{amssymb}
\usepackage[top=2cm,bottom=2cm]{geometry}
\usepackage{enumerate}
\usepackage{tikz}
\usepackage{amsfonts}
\title{Homework 3}
\author{Chengyu Lin\footnote{F1003028-5100309007}}
\date{}
\begin{document}
\maketitle

\begin{itemize}
    \item[Problem 1]
        That means choose two permutation from $[6]$ to fill two rows.
        And satisfies that each column has no repeated number.
        The answer equals $D_6 \times 6! = 190800$.

    \item[Problem 2]
        \begin{align*}
            S(n, n-2) &= {n \choose n-2} (\frac{n-2}{3} + \frac{(n-2)(n-3)}{4}) \\
                      &= {n \choose 3} + 3 {n \choose 4} \\
            S(n, n-3) &= {n \choose n-3} (\frac{n-3}{4} + \frac{3(n-3)(n-4)}{6} + \frac{(n-3)(n-4)(n-5)}{8}) \\
                      &= {n \choose 4} + 10{n \choose 5} + 15{n \choose 6} \\
            c(n, n-2) &= {n \choose n-2} (\frac{n-2}{3} \times 2! + \frac{(n-2)(n-3)}{4}) \\
                      &= 2 {n \choose 3} + 3 {n \choose 4} \\
            c(n, n-3) &= {n \choose n-3} (\frac{n-3}{4} \times 3! + \frac{3(n-3)(n-4)}{6} \times 2! + \frac{(n-3)(n-4)(n-5)}{8}) \\
                      &= 6 {n \choose 4} + 20{n \choose 5} + 15{n \choose 6}
        \end{align*}

    \item[Problem 3]
        \textbf{Proof: } Pick out the cycle with largest size (if there is a tie, choose the one
        whose minimal item is less). Knowing that $2k < n$, this cycle has a size greater than 2.
        That means the cycle has an inverse which is not equal to itself.

        So here we can construct an transformation which always transform the 'largest' cycle of
        one permutation to its inverse and keep other cycles invariant. Easy to see this transformation
        induces many pairs of permutations which can be transformed into each other.

        Which implies $s(n, k)$ is an even number.

    \item[Problem 4]
        $$\text{Ans} = \sum_{i=0}^{n}(-2)^i (2n - 2i)! {n \choose i} i! {2n - i + 1 \choose i}$$

        By PIE, first choose $i$ pairs whose sum is $2n + 1$ to put them together.
        Arrange the rest $2n - 2i$ items and these $i$ pairs. After put these $i$ pairs
        into items rest which has ${2n - 2i + i + 1 \choose i}$ ways.

        Or in an easy way:
        Just arrange the $2n - 2i + i$ items.

        $$\text{Ans} = \sum_{i=0}^{n}(-2)^i (2n - i)! {n \choose i}$$

    \item[Problem 5]

        First, by assign all $a_i = 1$ we can see that $a_1a_2a_3a_4 + a_2a_3a_4a_5 + \cdots + a_na_1a_2a_3 = n$.
        
        Each time we change the sign of one $a_i$, it will change the sign of 4 items in the formula above.
        And the sum of these 4 items will be 0, 2, -2, 4 or -4. After change, if the original sum was 2 it will be
        -2 and vice vesa, so do 4 and -4. So it will always makes a delta of 0, 4 or 8. Which implies the possible value
        of the formular is in $\{n - 4k | k \in \mathbb{N}\}$. When the formular equals zero, $n = 4k$.

    \item[Problem 6]

        Define $f(A) = \sum_{i=1}^{n} \sum_{j=1}^{n} A_{ij}$ which is the sum of every entries.
        When there comes a matrix A, if the sum of some lines were negative, just change the sign
        of every entries in this line. After manipulation $f(A)$ will strictly increase.

        And $f(A)$ has only a limited number of values (at most $2^{n*n}$). That means such a process would end in 
        finite steps, and every lines has a sum of non-negative number.

\end{itemize}

\end{document}
