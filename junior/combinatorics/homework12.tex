\documentclass{article}
\usepackage{amsmath}
\usepackage{amssymb}
\usepackage[top=2cm,bottom=2cm]{geometry}
\usepackage{enumerate}
\usepackage{tikz}
\usepackage{amsfonts}
\title{Homework 12}
\author{Chengyu Lin\footnote{F1003028-5100309007}}
\date{}
\begin{document}
\maketitle

\begin{itemize}
    \item[Problem 1]
        First, we have discussed in the class that the maximum
        size of this intersecting family is $2^{n-1}$ by pair
        each set and its complement.

        Suppose that there exists a intersecting family $\mathcal{F}$
        with size $2^{n-1}$ such that for some $A \subset B$, 
        $A \in \mathcal{F}$ but $B \not\in \mathcal{F}$.
        Since $A \subset B$, every set intersects $A$ that intersects
        $B$, $\mathcal{F} \cup \{B\}$ is also an intersecting
        family but its size would be greater than $2^{n-1}$,
        which is impossible. \#

    \item[Problem 2]
        ${10 \choose 3} = 120$, take every set from $2^{[12]}$
        which contains $1$ and $2$.

    \item[Problem 3]
        Take 5 sets with size 4: $\{1,2,3,4\}, \{2,3,4,5\},
        \{3,4,5,1\}, \{4,5,1,2\}, \{5,1,2,3\}$. The intersection
        between any two of them has size 3. Then there are
        6 elements remained in $[11]$ and $5 \times 6=30$.
        And with $\{1,2,3,4,5\}$ there are 31 sets.

    \item[Problem 4]
        $f(n) = {n \choose n/2}$

        Proof:

        For every subset $A$, denote $A_+ = \{\text{all positive
        elements lie in }A\}$, $A_- = A \setminus A_+$.

        If there are two distinct set $A$ and $B$,
        $A_+ \subseteq B_+$, $B_- \subseteq A_-$ infers that
        $|S(A) - S(B)| \ge 1$ (since every number has absolute
        value greater than 1).

        Define $A \preceq B$ iff $A_+ \subseteq B_+$ and 
        $B_- \subseteq A_-$. This partial order is equivalent to
        $\subseteq$ in $2^{[n]}$ (If $A$ contains positive $a_i$
        then set the $i$-th bit as 1 otherwise 0. If $A$ contains
        negative $a_i$ then set the $i$-th bit as 0 otherwise 1.)

        Clearly $\{A_m\}$ we choose must form an anti-chain in this
        poset, and we know that the maximum anti-chain size in $2^{[n]}$
        is ${n \choose n/2}$. \#

        Example:

        Let $a_i = 1$. Choose all sets of size $n/2$.
\end{itemize}

\end{document}
