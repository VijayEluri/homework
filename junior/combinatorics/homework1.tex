\documentclass{article}
\usepackage{amsmath}
\usepackage{amssymb}
\usepackage[top=2cm,bottom=2cm]{geometry}
\usepackage{enumerate}
\title{Homework 1}
\author{Chengyu Lin\footnote{F1003028-5100309007}}
\date{}
\begin{document}
\maketitle

\begin{itemize}
    \item[Problem 1]
        \begin{enumerate}[(a)]
            \item
                $${10\choose8} = {10\choose2} = \frac{10 \times 9}{2 \times 1} = 45$$
            \item
                The inverse of $1, 2, 3, 4, 5, 6$ in $\mathbb{Z}_7$ are
                $1, 4, 5, 2, 3, 6$.
            \item
                Easily that ${449 \choose 137} \equiv 0 (\text{mod }2)$ and
                ${449 \choose 137} \equiv 3 (\text{mod } 5)$.

                By Chinese Remainder Theorem ${449 \choose 137} \equiv 8 (\text{mod } 10)$.
        \end{enumerate}
    \item[Problem 2]
        For every ordered pair $(A, B)$ which $A \cap B = \emptyset$, 
        each element $x \in [n]$ has one of these
        three states: $x \in A$, $x \in B$ and $x \in [n] - A - B$.

        And clearly that the map between ordered pairs and state vector
        of elements is bijective.

        So the total number of ordered pairs is $3^n$.

    \item[Problem 3]
        %$$\sum_{r=0}^{n}r^2{n \choose r} = \sum_{A \subseteq [n]}|A|^2$$
        \begin{align*}
            (x + y)^n &= \sum_{r=0}^{n} {n \choose r}x^r y^{n-r} \\
            \frac{d(x + y)^n}{dx} &= \sum_{r=0}^{n} r {n \choose r}x^{r-1} y^{n-r} &= n(x + y)^{n-1} \\
            \frac{d(nx(x+y)^{n-1})}{dx} &= \sum_{r=0}^{n} r^2 {n \choose r} x^{r-1} y^{n-r} &= n(x+y)^{n-1} + n(n-1)x(x + y)^{n-2} \\
        \end{align*}

        Let $x = y = 1$, $\sum_{r=0}^{n} r^2 {n \choose r} = n2^{n-1} + n(n-1)2^{n-2} = n(n + 1)2^{n-2}$

        \textit{Combinatorial proof: }

        $\sum_{r=0}^{n}r^2 {n \choose r}$ means choose $r$ elements from $[n]$ and then
        pick 2 special elements (not necessary distinct) from these $r$ elements. That is
        the number of triples $\{(a, b), A\}$ where $A \subseteq [n]$ and $a, b \in A$.

        There is another way to count the number above: If $a = b$, the number of ways to pick $a$ and
        then determine whether the other element is in $A$ is $n2^{n-1}$. Otherwise the
        number of ways to pick $a$ and $b$ and then determine whether the other element is in $A$ is 
        $n(n-1)2^{n-2}$. Therefore the total number of ways to count it is $n(n+1)2^{n-2}$.


    \item[Problem 4]
        \begin{enumerate}[(a)]
            \item
                Let $i$ denotes the size of the blue square.

                $$\sum_{i=1}^{9} (9 - i + 1)^2 = 285$$
            \item
                First, count the number of disjoint squares which can be
                divide via a horizontal(or vertical) line ($i$ enumerates the line,
                $j$ enumerates the size of square):
                
                $$2 \times \sum_{i=1}^{8} ((\sum_{j=1}^{i} (9-j+1)(i-j+1)) \times (\sum_{j=1}^{9-i} 9-j+1)) = 37752$$

                Then count the number of disjoint squares which can be divice
                via a horizontal line and a vertical line.

                $$2 \times \sum_{i=1}^{8} \sum_{j=1}^{8} ((\sum_{k=1}^{min(i, j)} (i-k+1)\times(j-k+1)) 
                \times \min(9-i, 9-j)) = 11748$$

                So the total number is $2 \times (37752 - 11748) = 52008$.

        \end{enumerate}

    \item[Problem 5]
        Fill the 81 entries one-by-one using the ascending sequence from 1 to 81.
        Consider the first time every row or every column are filled with at least
        one number. WLOG, let's discuss with the first situation, and the last number
        placed into the entries is $x$.

        Clearly every row has at least one empty entry which is adjcent to the one
        was filled before. Otherwize there is some rows full-filled, that is the situation
        that every column are filled with at least one number. So there are at least
        9 empty entries which is adjcent to an entry filled with number less or equal
        than $x$. And there is no way to put the numbers left without violating the 
        condition given.

\end{itemize}

\end{document}
