\documentclass{article}
\usepackage{amsmath}
\usepackage{amssymb}
\usepackage[top=2cm,bottom=2cm]{geometry}
\usepackage{enumerate}
\usepackage{tikz}
\usepackage{amsfonts}
\title{Homework 9}
\author{Chengyu Lin\footnote{F1003028-5100309007}}
\date{}
\begin{document}
\maketitle

\begin{itemize}
    \item[Problem 1]
        \begin{enumerate}[(a)]
            \item
                $$E(X_v) = \frac{{6 \choose 3} \times 2 \times 3!}{6!} = \frac{1}{3}$$

                $$E(X) = \sum_{v} E(X_v) = 2$$

                And no matter what $\sigma$ is, each triangle has one and only one seed.
                Therefore the probability is $1$.

            \item
                $E(X)$ is the same as above.

                Fix the position of 1. 
                If another seed is 2, considering the position of 2
                there are $4! + 2(4! - 3! - 2)$ outcomes. Otherwise 2 is
                adjacent to 1.
                If another seed is 3, there are $2(3! \times 2)$ outcomes.
                Otherwise 3 is adjacent to 2 or 1.
                If another seed is 4, there are $2 \times 2 \times 2$ outcomes.

                So $Pr(X = 2) = \frac{4! + 2(4! - 3! - 2) + 2(3! + 2) + 2 \times 2 \times 2}{5!} = \frac{11}{15}$.

        \end{enumerate}

    \item[Problem 2]
        For odd $n$, considering two coloring in $C_n$, events ${E_i}$ are
        whether one edge is not monochromatic. 
        Then $Pr(\cap_{i \in S}E_i) = \frac{1}{2^{|S|}} (S \neq [n])$,
        but $Pr(\cap_{i \in [n]}E_i) = 0$.

        Otherwise provide $n - 1$ random variables ${a_i}$.
        The first $n - 1$ events show that $a_i = 1$.
        And the $n$-th event shows $\sum_{i = 1}^{n - 1}a_i \equiv 1 (mod 2)$

        Then $Pr(\cap_{i \in S} E_i = \frac{1}{2^{|S|}}) (S \subseteq [n-1])$.

            $Pr(\cap_{i \in S} E_i \cap E_n) = \frac{1}{2^{|S| + 1}} (S \subset [n - 1])$, 

            But $Pr(\cap_{i \in [n]}E_i) = \frac{1}{2^{n-1}}$.
    \item[Problem 3]
        Color the vertices uniformly at random.

        Events: for each edge $(u, v)$ where $S(u) \cap S(v) \neq \emptyset$, one
        event denotes that $u$ and $v$ are colored with $c$. At most $|V| \times 10 d^2$ events.

        Dependency: One edge is conditional independent to another set of edges
        if the two vertices is not adjacent to any edge of that set.
        Because the probability for one edge $(u, v)$ to be monochromatic is 
        $p = Pr(S_u = c)Pr(S_v = c) = \frac{1}{100d^2}$, 
        no matter what's the color of other vertices.

        Degrees: So in the dependency graph,
        for each events there are at most $20d^2$ events adjacent to it.
        Since each vertex has at most $d$ vertices for one color so that
        the edge between them can be monochromatic.

        $4 \times \frac{1}{100d^2} \times 20d^2 = \frac{4}{5} < 1$.
        By L. L. L., $Pr(\text{every edge is not monochromatic}) > 0$.
        So there exists an outcome which is a proper colouring.

    \item[Problem 4]
        \begin{enumerate}[(a)]
            \item
                Since every $Z_a$ has fixed length $k$,
                the common difference will be no greater than $\frac{n}{k - 1}$.

                And for every number, it could lie in at most $\frac{n}{k - 1}k$
                a.p.s with length k since a.p. is uniquely determined by a starter
                and a common difference.

                So for any $Z_a$, there are at most $\frac{n}{k - 1} k^2$ $Z_a$s
                can have non-empty intersection with it. Which is less than
                $1.25kn$ when $k \ge 10$.
            \item
                Color every elements uniformly at random.

                Events: for each $Z_a$ an event shows whether it's monochromatic.
                The probability for one event to be true is $2^{1-k}$.

                Dependency: One event $Z_a$ is independent to another set of $Z_a$s
                if this $Z_a$ do not share any points with that set.
                In dependency graph one event has degree less than $1.25kn = 2^{k-3}$.

                $4 p d < 4 \times 2^{1-k} \times 2^{k-3} = 1$.
                By L. L. L., $Pr(\text{every $Z_a$ is not monochromatic}) > 0$.
                So there exists an outcome satisfies the condition.
        \end{enumerate}

\end{itemize}

\end{document}
