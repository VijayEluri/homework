\documentclass{article}
\usepackage{amsmath}
\usepackage{amssymb}
\usepackage[top=1cm,bottom=1cm]{geometry}
\usepackage{enumerate}
\title{Homework 1}
\author{Chengyu Lin\footnote{F1003028-5100309007}}
\date{}
\begin{document}
\maketitle

\begin{itemize}
    \item[Problem 1]
        It's easy to come out a polynomial-time algorithm
        related to the number of types and the value of each coins.

        For the polynomial-time algorithm considering only the number of
        types, refer to
        \textit{A Polynomial-time Algorithm for the Change-Making Problem}
        [David Pearson, 1994].

    \item[Problem 2]
        "$\Rightarrow$": 

        Assume that the greedy algorithm outputs $S$. And one of the
        maximum cost set is $T$.

        First I'm going to show that $|S| = |T|$. Clearly that 
        $|S| \leq |T|$, otherwise we can find some $t \in S-T$. And
        according to the definition of matroid, $T \cup \{t\}$ is
        a independent set and provide more cost than $T$, which contradicts
        the assumption that $T$ is one of the maximum cost set.
        On the other hand $|T| \leq |S|$, otherwise there is some 
        $t \in T - S$ satisfied $S \cup \{t\}$ is a independent set.
        And according to the step 3.1 of the algorithm $t$ will be one of 
        the elements in $S$. Therefore $|S| = |T|$.

        Now let $S = \{a_{i_1}, a_{i_2}, \cdots, a_{i_m}\}$
        and $T = \{a_{j_1}, a_{j_2}, \cdots, a_{j_m}\}$
        , $\{i_m\}$ and $\{j_m\}$ are both ascending sequences ranged from
        1 to $n$.
        Assume $T$ is the maximum cost set which maximized $k$ where 
        $i_p = j_p (p \leq k)$.

        If $k = m$ then it's done.
        When $k < m$, $i_{k + 1}$ must be less than $j_{k + 1}$ otherwise
        the greedy algorithm will choose $a_{j_{k + 1}}$ first. Now add
        $a_{i_{k + 1}}$ to $T$, and kick one element from 
        $\{a_{j_{k + 1}}, \cdots, a_{j_m}\}$ to form a independent set $A$.
        Clearly that the cost of $A$ is not small than $T$. But the $k$
        for $A$ where $i_p = j_p (p \leq k)$ is larger than $T$, which
        contradicts the assumption.

        Therefore the greedy algorithm always gives the maximum cost set.

        "$\Leftarrow$": 

        I'm going to show that if $\{V, \mathbf{I}\}$ is not 
        a matroid, then the greedy algorithm would fail sometimes.

        Assume that $V = \{1, 2, 3\}$ and 
        $\mathbf{I} = \{\emptyset, \{1\}, \{2\}, \{3\}, \{2, 3\}\}$. 
        The cost function is $c(1) = 3$, $c(2) = c(3) = 2$.
        The greedy algorithm would provides $\{1\}$ but $\{2, 3\}$ is
        a better solution.

    \item[Problem 3]
        Proof by induction:

        After the first enumeration, 
        the algorithm provides $\{a_1\}$ which is the optimal
        solution for $\{a_1\}$.

        Assume after enumerating $a_{k - 1}$ 
        (before $a_k$ is considered), the algorithm
        provides the optimal soluton $S$ 
        for $\{a_1, a_2, \cdots, a_{k - 1}\}$ which is the same as
        offline greedy algorithm provides.

        If $S \cup \{a_k\} \in \mathbb{I}$, clearly that
        $S \cup \{a_k\}$ is the optimal solution for $\{a_1, \cdots, a_k\}$.

        WLOG, the first $k-1$ elements are sorted by their cost.
        When we inserted $a_k$, assume that for all $i \leq j$, 
        $a_k <= a_i$ and for all $i > j$, $a_k > a_i$.
        Now let's perform offline algorithm on that.
        For the first $j$ elements, both online and offline algorithm
        will give the same result. After that if $a_k$ contradicts the 
        previous choice then it's done. Else we will put $a_k$ into the
        result. Then for the rest elements we will choose the elements
        by greedy, and it will be almost the same as the optimal solution 
        for first $k-1$ elements except that there maybe one element can
        not be chosen since we have add $a_k$ before. 
        And clearly that this process will give the same result as the
        online algorithm does (step 3).

        Therefore this online algorithm will provides the same result
        for the first $k$ elements as offline algorithm does.

        So by induction it will gives the optimal solution.

\end{itemize}

\end{document}
